\documentclass{article}
\usepackage[ngerman]{babel}
\usepackage[left=2.5cm, right=2.5cm, top=2.5cm, bottom=2cm]{geometry}
\usepackage[onehalfspacing]{setspace}
\usepackage{booktabs}
\usepackage{float}

\usepackage{listings}
\usepackage{xcolor}

\definecolor{codegreen}{rgb}{0,0.6,0}
\definecolor{codegray}{rgb}{0.5,0.5,0.5}
\definecolor{codepurple}{rgb}{0.58,0,0.82}
\definecolor{backcolour}{rgb}{0.95,0.95,0.92}

\lstdefinestyle{mystyle}{
    backgroundcolor=\color{backcolour},   
    commentstyle=\color{codegreen},
    keywordstyle=\color{magenta},
    numberstyle=\tiny\color{codegray},
    stringstyle=\color{codepurple},
    basicstyle=\ttfamily\footnotesize,
    breakatwhitespace=false,         
    breaklines=true,                 
    captionpos=b,                    
    keepspaces=true,                 
    numbers=left,                    
    numbersep=5pt,                  
    showspaces=false,                
    showstringspaces=false,
    showtabs=false,                  
    tabsize=2,
    language=[LaTeX]{TeX}
}
\lstset{style=mystyle}

\begin{document}
\title{Mini-\LaTeX-Doku}
\author{Max Schmidt}
\date{heute oder so}

\maketitle
\thispagestyle{empty}

\tableofcontents
\thispagestyle{empty}
\newpage

\section{\LaTeX\ Editor Setup}
Für Windoofs: Die drei Sachen herunterladen und installieren: \\
https://code.visualstudio.com/, https://miktex.org/download, https://strawberryperl.com/ \\
In VS-Code die Extension LaTeX Workshop installieren. MikTex Console öffnen (wenn es nicht geht dann schließen und wieder öffen).
In der MikTex Console unter Updates Check for Updates und dann Update now. Fertig \\
Wenn du Linux verwendest kannst dus selber besser :) \\
Wenn du Apfel verwendest ist dir nicht mehr zu helfen.

\section{Ein Dokument erstellen}
    Um ein Dokument zu erstellen einfach einen neuen Ordner erstellen und dann in VS Code
    eine neue Datei mit der Endung .tex erstellen.
    Folgender Code ist eigentlich immer ein guter Start:
    \begin{lstlisting}
    \documentclass{article}
    \usepackage[ngerman]{babel} %damit von LaTeX erstellte Texte auf Deutsch sind
    \usepackage[left=2.5cm, right=2.5cm, top=2.5cm, bottom=2cm]{geometry} %Seitenraender
    \usepackage[onehalfspacing]{setspace} %Zeilenabstaende

    \begin{document}
    Hier koennte Ihre Werbung stehen!
    \end{document}
    \end{lstlisting}
    Mit dem günen Pfeil oben rechts in VS Code kann das Dokument das erste mal compiled werden.
    Ab dann wird nach dem Speichern (Strg + S) automatisch immer wieder compiled und die PDF neu erstellt.
        
\section{Titelseite und Inhaltsverzeichnis}
    Damit \LaTeX weiß was auf der Titelseite Stehen soll muss das zuerst definiert werden: Es kann alles in die Klammern geschrieben werden.
    \begin{lstlisting}
    \title{Titel des Dokuments}
    \author{Max Schmidt}
    \date{Juli 23}
    \end{lstlisting}
    Um die Titelseite dann zu erstellen:
    \begin{lstlisting}
    \maketitle
    \thispagestyle{empty} %wenn auf der Titelseite keine Seitenzahl stehen soll
    \newpage %fuer einen Seitenumbruch
    \end{lstlisting}
    Um das Inhaltsverzeichnis zu erstellen:
    \begin{lstlisting}
    \tableofcontents
    \thispagestyle{empty}  %wenn auf der Seite keine Seitenzahl stehen soll
    \newpage %fuer einen Seitenumbruch
        \end{lstlisting}
\section{Text Formatierung}
\LaTeX\ ist kein \glqq What you see is what you get\grqq also Hilft es dir nix im Code Zeilenumbrüche zu machen.

\begin{figure}[H]
    \centering
    \begin{tabular}{l|l}
        \textbf{Formatierung} & \textbf{Beispielcode} \\
        \midrule
        Fett & \lstinline|\textbf{Fetter Text}| \\
        Kursiv & \lstinline|\textit{Kursiver Text}| \\
        Unterstrichen & \lstinline|\underline{Unterstrichener Text}| \\
        Durchgestrichen & \lstinline|\sout{Durchgestrichener Text}| \\
        Schriftgröße & \lstinline|\tiny, \scriptsize, \footnotesize,| \\
        & \lstinline|\small, \normalsize, \large,| \\
        & \lstinline|\Large, \huge, \Huge| \\
        Schriftart & \lstinline|\textsf{Serifenlose Schrift}| \\
        & \lstinline|\texttt{Monospaced Schrift}| \\
        Schriftfarbe & \lstinline|\textcolor{Farbe}{Text}| \\
        Hoch- / Tiefstellung & \lstinline|\textsuperscript{Hochgestellter Text}| \\
        & \lstinline|\textsubscript{Tiefgestellter Text}| \\
        Zeilenumbruch & \lstinline|\\| oder \lstinline|\newline| \\
        \end{tabular}
    \centering
    
    
\end{figure}
Sollte nach einem Zeilenumbruch ein komischer Einzug entstehen kann man dem mit \lstinline|\noindent| entgehen.
    
\section{Sections}
\section{Tabellen, Listen}
\section{Abblidungen}
\section{Mathematische Formeln}
\section{Code Snipets}


\end{document}